\documentclass[12pt,a4paper,utf8]{ctexart}
\usepackage{graphicx}
\usepackage{amsmath}
\usepackage{amssymb}
\usepackage{subfig}
\usepackage{cite}
\usepackage[ntheorem]{empheq}
\usepackage{enumitem}
\usepackage{fullpage}
\usepackage{cleveref}
\usepackage{cellspace}
\usepackage{listings}
\usepackage{color}
\definecolor{gray}{rgb}{0.5,0.5,0.5}
\definecolor{dkgreen}{rgb}{.068,.578,.068}
\definecolor{dkpurple}{rgb}{.320,.064,.680}

% set Matlab styles
\lstset{
   language=Matlab,
   keywords={break,case,catch,continue,else,elseif,end,for,function,
      global,if,otherwise,persistent,return,switch,try,while},
   basicstyle=\ttfamily,
   keywordstyle=\color{blue}\bfseries,
   commentstyle=\color{dkgreen},
   stringstyle=\color{dkpurple},
   backgroundcolor=\color{white},
   tabsize=4,
   showspaces=false,
   showstringspaces=false
}

\begin{document}
\CJKfamily{zhkai}	


\begin{center}
\textbf{作业一}\\
\textbf{姓名 ~~~~~~~~~~~~~ 学号 ~~~~~~~~~~~~~~ 日期}\\
\end{center}

\begin{center}
\fbox{
\begin{minipage}{40em}
\vspace{5cm}
\hspace{20cm}
\end{minipage}}
\end{center}
\vspace{1cm}

\begin{enumerate}
\item[第一题] \textbf{重心插值公式(barycentric interpolation formula)}  

课堂上我们已经讨论过了基于$n+1$个插值点$\{x_j\}_{j=0}^n$的Lagrange插值多项式:
\begin{equation}
p(x) = \sum_{j=0}^{n} f_j \ell_j(x) \label{lagrange}
\end{equation}
此处,$f_j = f(x_j)$。Lagrange插值基函数(Lagrange polynomial)
\begin{equation}
\ell_{j}(x)=\frac{\prod_{k \neq j}\left(x-x_{k}\right)}{\prod_{k \neq j}\left(x_{j}-x_{k}\right)} \label{cardinal}
\end{equation}
满足
\begin{equation}
\ell_{j}\left(x_{k}\right)=\left\{\begin{array}{ll}
1 & k=j \\
0 & k \neq j
\end{array}\right. \nonumber
\end{equation}


\item[第二题]
\textsc{Matlab}程序显示如下:
\begin{lstlisting}[frame=single]
a = -1;
op = @(x) 1./((x-a).^2);
p = 4.5e-16;
y = a+p;
val = op(y)
fvals = f(1-x);
while( all(fvals(2:end)./fvals(1:end-1) > testRatio) )
    poleOrder = poleOrder + 1;
    fvals = fvals.*x;
end
\end{lstlisting}

\item[第三题]



\end{enumerate}




\end{document}
