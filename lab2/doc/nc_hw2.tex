\documentclass[12pt,a4paper,UTF8]{ctexart}
\usepackage{graphicx}
\usepackage{amsmath}
\usepackage{amssymb}
\usepackage{cite}
\usepackage[ntheorem]{empheq}
\usepackage{enumitem}
\usepackage{fullpage}
\usepackage{tocbibind}
\usepackage[bookmarksopen=true,colorlinks,linkcolor=black]{hyperref}
\usepackage{cellspace}
\usepackage{listings}
\usepackage{color}
\usepackage{epstopdf}
\usepackage{subfigure}
\usepackage{algorithm}
\usepackage{algorithmicx}
\usepackage{algpseudocode}
\usepackage{lipsum}
\usepackage[thmmarks,amsmath]{ntheorem}



\theoremstyle{nonumberplain}

\theoremheaderfont{\bfseries}

\theorembodyfont{\normalfont}

\theoremsymbol{$\square$}

\newtheorem{Proof}{\hskip 2em 证明}


\makeatletter
\newenvironment{breakablealgorithm}
  {% \begin{breakablealgorithm}
   \begin{center}
     \refstepcounter{algorithm}% New algorithm
     \hrule height.8pt depth0pt \kern2pt% \@fs@pre for \@fs@ruled
     \renewcommand{\caption}[2][\relax]{% Make a new \caption
       {\raggedright\textbf{\ALG@name~\thealgorithm} ##2\par}%
       \ifx\relax##1\relax % #1 is \relax
         \addcontentsline{loa}{algorithm}{\protect\numberline{\thealgorithm}##2}%
       \else % #1 is not \relax
         \addcontentsline{loa}{algorithm}{\protect\numberline{\thealgorithm}##1}%
       \fi
       \kern2pt\hrule\kern2pt
     }
  }{% \end{breakablealgorithm}
     \kern2pt\hrule\relax% \@fs@post for \@fs@ruled
   \end{center}
  }
\makeatother
\renewcommand{\algorithmicrequire}{\textbf{Input:}}  % Use Input in the format of Algorithm
\renewcommand{\algorithmicensure}{\textbf{Output:}} % Use Output in the format of Algorithm
\usepackage{longtable}

\usepackage{float}
\definecolor{gray}{rgb}{0.5,0.5,0.5}
\definecolor{dkgreen}{rgb}{.068,.578,.068}
\definecolor{dkpurple}{rgb}{.320,.064,.680}

% set Matlab styles
\lstset{
   language=Matlab,
   numbers=left,
   keywords={break,case,catch,continue,else,elseif,end,for,function,
      global,if,otherwise,persistent,return,switch,try,while},
   basicstyle=\ttfamily,
   keywordstyle=\color{blue}\bfseries,
   commentstyle=\color{dkgreen},
   stringstyle=\color{dkpurple},
   backgroundcolor=\color{white},
   breaklines=true,
   tabsize=4,
   showspaces=false,
   showstringspaces=false,
}

\begin{document}
\CJKfamily{zhkai}


\begin{center}
    \textbf{作业二}\\
    \textbf{姓名 胡毅翔 ~~ 学号 PB18000290 ~~ 日期 2021年5月29日}\\
\end{center}

\begin{center}
    \fbox{
        \begin{minipage}{40em}
            \vspace{5cm}
            \hspace{20cm}
        \end{minipage}}
\end{center}
\vspace{1cm}

\begin{enumerate}
    \item[第一题] 本题考虑对于定义在 $[-1,1]$ 上的一个光滑函数 $f(x)$ 的三次样条插值的使用。下面 所说的误差都是指绝对误差。
    \begin{enumerate}\item(10分)仿照课堂笔记或课本推导出关于额外给定边界点处(即-1和1)三
    次样条插值多项式的一次导数值时其在各插值点上的二次导数值应该满足的
    线性方程组。请给出推导过程。
    \par 解:
    \par 记 $S(x)$ 在区间 $\left[x_{i}, x_{i+1}\right]$ 上的表达式为 $S_{i}(x), S(x)$ 是三次多项式, $S^{\prime \prime}(x)$ 是 线性函数, 用插值点 $\left\{\left(x_{i}, S^{\prime \prime}\left(x_{i}\right)\right),\left(x_{i+1}, S^{\prime \prime}\left(x_{i+1}\right)\right)\right\}$ 作线性插值, 记 $S^{\prime \prime}\left(x_{i}\right)=M_{i}$,
    $S^{\prime \prime}\left(x_{i+1}\right)=M_{i+1}$
    $$
    S_{i}^{\prime \prime}(x)=\frac{x-x_{i+1}}{x_{i}-x_{i+1}} M_{i}+\frac{x-x_{i}}{x_{i+1}-x_{i}} M_{i+1}, \quad x_{i} \leqslant x \leqslant x_{i+1}
    $$
    对 $S^{\prime \prime}(x)$ 积分两次, 记 $h_{i}=x_{i+1}-x_{i}$,
    $$
    \begin{aligned}
    S(x) &=S_{i}(x)=\frac{\left(x_{i+1}-x\right)^{3}}{6 h_{i}} M_{i}+\frac{\left(x-x_{i}\right)^{3}}{6 h_{i}} M_{i+1}+c x+d \\
    &=\frac{\left(x_{i+1}-x\right)^{3}}{6 h_{i}} M_{i}+\frac{\left(x-x_{i}\right)^{3}}{6 h_{i}} M_{i+1}+C\left(x_{i+1}-x\right)+D\left(x-x_{i}\right)
    \end{aligned}
    $$
    将 $S\left(x_{i}\right)=y_{i}, S\left(x_{i+1}\right)=y_{i+1}$ 代入上式解出
    $$
    C=\frac{y_{i}}{h_{i}}-\frac{h_{i} M_{i}}{6}, \quad D=\frac{y_{i+1}}{h_{i}}-\frac{h_{i} M_{i+1}}{6}
    $$
    \begin{equation}\label{eq1}
    \begin{aligned}
S(x)=& \frac{\left(x_{i+1}-x\right)^{3} M_{i}+\left(x-x_{i}\right)^{3} M_{i+1}}{6 h_{i}}+\frac{\left(x_{i+1}-x\right) y_{i}+\left(x-x_{i}\right) y_{i+1}}{h_{i}} \\
&-\frac{h_{i}}{6}\left[\left(x_{i+1}-x\right) M_{i}+\left(x-x_{i}\right) M_{i+1}\right], \quad x \in\left[x_{i}, x_{i+1}\right]
\end{aligned}\end{equation}

在内结点 $x_{i}$, 由 $S_{i}^{\prime}\left(x_{i}\right)=S_{i-1}^{\prime}\left(x_{i}\right)$ 可得到
\begin{equation}\label{eq2}
f\left(x_{i}, x_{i+1}\right)-\frac{h_{i}}{3} M_{i}-\frac{h_{i}}{6} M_{i+1}=f\left(x_{i-1}, x_{i}\right)+\frac{h_{i-1}}{6} M_{i-1}+\frac{h_{i-1}}{3} M_{i}
\end{equation}
整理后得到
$$
\mu_{i} M_{i-1}+2 M_{i}+\lambda_{i} M_{i+1}=d_{i}, \quad i=1,2, \cdots, n-1
$$
其中
$$
\begin{array}{c}
\lambda_{i}=\frac{h_{i}}{h_{i}+h_{i-1}}, \quad \mu_{i}=1-\lambda_{i} \\
d_{i}=\frac{6}{h_{i}+h_{i-1}}\left(\frac{y_{i+1}-y_{i}}{h_{i}}-\frac{y_{i}-y_{i-1}}{h_{i-1}}\right)=6 f\left(x_{i-1}, x_{i}, x_{i+1}\right)
\end{array}
$$
式 \ref{eq2} 称为样条插值的 $M$ 关系方程组,解方程组 \ref{eq2} 得到 $\left\{M_{i}, i=\right.$ $\left.1,2, \cdots, M_{n-1}\right\}$, 再加上两个端点条件, 满足端点条件的样条插值函数 $S(x)$ 在 $\left[x_{i},\right.$, $\left.x_{i+1}\right]$ 上的表达就是式 \ref{eq1}.
\par 给定 $S^{\prime}\left(x_{0}\right)=m_{0}, S^{\prime}\left(x_{n}\right)=m_{n}$ 的值, 将 $S^{\prime}\left(x_{0}\right)=m_{0}, S^{\prime}\left(x_{n}\right)=m_{n}$ 的值分
别代入 $S^{\prime}(x)$ 在 $\left[x_{0}, x_{1}\right],\left[x_{n-1}, x_{n}\right]$ 中的表达式, 得到另外两个方程:
$$
\begin{array}{c}
2 M_{0}+M_{1}=\frac{6}{h_{0}}\left[f\left[x_{0}, x_{1}\right]-m_{0}\right]=d_{0} \\
M_{n-1}+2 M_{n}=\frac{6}{h_{n-1}}\left[m_{n}-f\left[x_{n-1}, x_{n}\right]\right]=d_{n}
\end{array}
$$
得到 $n+1$ 个未知量, $n+1$ 个方程组
$$
\left[\begin{array}{cccccc}
2 & 1 & & & & \\
\mu _{1} & 2 & \lambda_{1} & & & \\
& \mu _{2} & 2 & \lambda_{2} & & \\
& & \ddots & \ddots & \ddots & \\
& & & \mu _{n-2} & 2 & \lambda_{n-1} \\
& & & & 1 & 2
\end{array}\right]\left[\begin{array}{c}
M_{0} \\
M_{1} \\
M_{2} \\
\vdots \\
M_{n-1} \\
M_{n}
\end{array}\right]=\left[\begin{array}{c}
d_{0} \\
d_{1} \\
d_{2} \\
\vdots \\
d_{n-1} \\
d_{n}
\end{array}\right]
$$
\item(10分)令三次样条插值多项式在一1和1处的导数为0, 用\textbf{Matlab}基于上
    一问中的结果使用 $n=2^{4}$ 个子区间插值一个定义在 $[-1,1]$ 上的函数 $f(x)=$ $\sin \left(4 x^{2}\right)+\sin ^{2}(4 x)$ 并使用semilogy图通过在2000个等距点上取真实值画出你 构造的三次样条插值的逐点误差。
    \item(15分)使用不同的 $n$, 令 $n=2^{4}, 2^{5}, \ldots, 2^{10}$ 重复上一问,取关于不同 $n$ 的2000个
    等距点上的误差的最大值,用loglog图描述插值区
    间上最大误差值随 $n$ 变化
    的情况(即横轴是 $n$ )。
    \item(15分)针对周期边界条件,即假设三次样条函数满足 $S^{\prime}(-1)=S^{\prime}(1)$ 和 $S^{\prime \prime}(-1)=$ $S^{\prime \prime}(1)$, 重复完成上面三问中的要求。
         

\end{enumerate}
\item[第二题] 本题深入讨论Newton插值公式的性质。
\begin{enumerate}\item(15分) 对于一个光滑函数 $f(x)$, 证明若 $\left\{i_{0}, i_{1}, \ldots, i_{k}\right\}$ 是 $\{0,1, \ldots, k\}$ 的任意一 个排列,则
$$
f\left[x_{0}, x_{1}, \ldots, x_{k}\right]=f\left[x_{i_{0}}, x_{i_{1}}, \ldots, x_{i_{k}}\right]
$$
\par 证:
\par 先证引理:$k$ 阶差商 $f\left[x_{0}, x_{1}, \cdots, x_{k}\right]$ 是由函数值 $f\left(x_{0}\right), f\left(x_{1}\right), \cdots, f\left(x_{k}\right)$的
线性组合而成.
$$
f\left[x_{0}, x_{1}, \cdots, x_{k}\right]=\sum_{i=0}^{k} \frac{1}{\left(x_{i}-x_{0}\right) \cdots\left(x_{i}-x_{i-1}\right)\left(x_{i}-x_{i+1}\right) \cdots\left(x_{i}-x_{k}\right)} f\left(x_{i}\right)
$$
用归纳法可以证明这一引理。
\par 显然,当$k=1$时,引理成立。
\par 假设当$k=n-1$时,引理成立,故有:
$$f\left[x_{0}, x_{1}, \cdots, x_{n-1}\right]=\sum_{i=0}^{n-1} \frac{1}{\left(x_{i}-x_{0}\right) \cdots\left(x_{i}-x_{i-1}\right)\left(x_{i}-x_{i+1}\right) \cdots\left(x_{i}-x_{n-1}\right)} f\left(x_{i}\right)$$
$$f\left[x_{1}, x_{2},\cdots, x_{n}\right]=\sum_{i=1}^{n} \frac{1}{\left(x_{i}-x_{1}\right) \cdots\left(x_{i}-x_{i-1}\right)\left(x_{i}-x_{i+1}\right) \cdots\left(x_{i}-x_{n}\right)} f\left(x_{i}\right)$$
\par 则$k=n$时:
$$
\begin{aligned}
  f\left[x_{0}, x_{1}, \cdots, x_{n}\right] & =\frac{f\left[x_{0}, x_{1}, \cdots, x_{n-1}\right]-f\left[x_{1}, x_{2},\cdots, x_{n}\right]}{x_{0}-x_{n}}\\
 & =\sum_{i=0}^{k} \frac{1}{\left(x_{i}-x_{0}\right) \cdots\left(x_{i}-x_{i-1}\right)\left(x_{i}-x_{i+1}\right) \cdots\left(x_{i}-x_{n}\right)} f\left(x_{i}\right)\end{aligned}
$$
\par 引理证毕,故有:
$$
f\left[x_{0}, x_{1}, \cdots, x_{k}\right]=\sum_{i=0}^{k} \frac{1}{\left(x_{i}-x_{0}\right) \cdots\left(x_{i}-x_{i-1}\right)\left(x_{i}-x_{i+1}\right) \cdots\left(x_{i}-x_{k}\right)} f\left(x_{i}\right)
$$
$$
f\left[x_{i_{0}}, x_{i_{1}}, \cdots, x_{i_{k}}\right]=\sum_{j=0}^{k} \frac{1}{\left(x_{i_{j}}-x_{i_{0}}\right) \cdots\left(x_{i_{j}}-x_{i_{j-1}}\right)\left(x_{i_{j}}-x_{i_{j+1}}\right) \cdots\left(x_{i_{j}}-x_{i_{k}}\right)} f\left(x_{i_{j}}\right)
$$
\par 易得:
$$
f\left[x_{0}, x_{1}, \ldots, x_{k}\right]=f\left[x_{i_{0}}, x_{i_{1}}, \ldots, x_{i_{k}}\right]
$$
\item (10分)课堂上我们提到了Chebyshev点
$$
x_{j}=\cos (j \pi / n) \quad j=0,1, \ldots, n
$$
以及使用Chebyshev点可以有效地克服Runge现象。写一个MATLAB程序,令 $n=2^{2}, 2^{3}, 2^{4}, \ldots, 2^{7}$, 按照从右到左的顺序(即 $j$ 从小到大的顺序)使用对应
的 $n+1$ 个Chebyshev点对定义在 $[-1,1]$ 上的Runge函数
$$
f(x)=\frac{1}{1+25 x^{2}}
$$
进行插值,并取2000个等距点上的误差的最大值,用semilogy图描述插值区
间上最大误差值随 $n$ 变化的情况(即横轴是 $n$ 。
\item (10分) 重复上一问,但使用随机数种子rng(22)和randperm函数来随机计算 差商时插值点的使用顺序,取关于不同 $n$ 的2000个等距点上的误差的最大值,
用semilogy图描述插值区间上最大误差值随 $n$ 变化的情况(即横轴是 $n$ 。
\item(10分)试着解释上面两小问中你观察到的不同现象产生的原因。注: 此问
答不出来也无妨。
\end{enumerate}
\item[第三题] 本题用于讨论周期函数的Lagrange插值方法。对于周期函数而言,多项式不再是 最有效的基函数,而等距插值点也不再会出现Runge现象。逼近周期函数的基函 数通常选用三角函数或者复指数。同时注意对于周期函数而言,插值点数量和子
区间个数相等。
\begin{enumerate}\item (10分)在 $[0,1]$ 上关于周期函数的基于等间距插值点 $x_{j}=\frac{j}{n}, j=0,1, \ldots$, $n-1$ 的Lagrange插值基函数为
$$
\ell_{k}(x)=\left\{\begin{array}{ll}
\frac{(-1)^{k}}{n} \sin (n \pi x) \csc \left(\pi\left(x-x_{k}\right)\right) & \text { 若 } n \text { 为奇数 } \\
\frac{(-1)^{k}}{n} \sin (n \pi x) \cot \left(\pi\left(x-x_{k}\right)\right) & \text { 若 } n \text { 为偶数 }
\end{array}\right.
$$
证明对于 $n$ 分别为奇数和偶数的情况下
$$
\ell_{k}\left(x_{j}\right)=\left\{\begin{array}{ll}
1 & k=j \\
0 & k \neq j
\end{array}\right.
$$
\item (10分)用上述对应于 $n$ 为偶数的Lagrange基函数构造Lagrange插值多项式. 并用 $n=2^{6}$ 个点对周期函数 $f(x)=\sin (2 \pi x) e^{\cos (2 \pi x)}$ 在 $[0,1]$ 上进行插值。取1000个 等距点上的误差,用semilogy图描述插值区间上误差值随 $x$ 变化的情况(即
横轴是 $x$ )。

\end{enumerate}
\item[第四题] (10分) 写程序完成课本 59 页第7题,并计算出你的拟合函数对比所给数据点的
误差的2-范数。
\par 解:
\par 本题的\textbf{MATLAB}程序显示如下:
\begin{lstlisting}[frame=single]
clc, clear
x = [2.1, 2.5, 2.8, 3.2];
y = [0.6087, 0.6849, 0.7368, 0.8111];
a_1 = 0;
b_1 = 0;
x_y = x .* y;
c_11 = sum(x .* x);
c_12 = -sum(x .* x_y);
c_21 = -c_12;
c_22 = -sum(x_y .* x_y);
B_1 = sum(x .* y);
B_2 = sum(x_y .* y);
C = [c_11, c_12; c_21, c_22];
B = [B_1; B_2];
tmp = C \ B;
a_1 = tmp(1);
b_1 = tmp(2);
a = 1 / a_1;
b = b_1 / a_1;
y_bar = [Phi(x(1), a, b), Phi(x(2), a, b), Phi(x(3), a, b), Phi(x(4), a, b)];
delta_y = y - y_bar;
X = sprintf('The 2-norm of the error of the fitting function compared to the given data point is %.15f', norm(delta_y, 2));
disp(X)

function value = Phi(x, a, b)
    value = x / (a + b * x);
end

\end{lstlisting}
\par 程序运行输出结果为:
\begin{lstlisting}[frame=single]
The 2-norm of the error of the fitting function compared to the given data point is 0.005738349475781
\end{lstlisting}
\end{enumerate}




\end{document}
