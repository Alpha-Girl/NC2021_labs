\documentclass[12pt,a4paper,UTF8]{ctexart}
\usepackage{graphicx}
\usepackage{amsmath}
\usepackage{amssymb}
\usepackage{cite}
\usepackage[ntheorem]{empheq}
\usepackage{enumitem}
\usepackage{fullpage}
\usepackage{tocbibind}
\usepackage[bookmarksopen=true,colorlinks,linkcolor=black]{hyperref}
\usepackage{cellspace}
\usepackage{listings}
\usepackage{color}
\usepackage{epstopdf}
\usepackage{subfigure}
\usepackage{algorithm}
\usepackage{algorithmicx}
\usepackage{algpseudocode}
\usepackage{lipsum}

\makeatletter
\newenvironment{breakablealgorithm}
  {% \begin{breakablealgorithm}
   \begin{center}
     \refstepcounter{algorithm}% New algorithm
     \hrule height.8pt depth0pt \kern2pt% \@fs@pre for \@fs@ruled
     \renewcommand{\caption}[2][\relax]{% Make a new \caption
       {\raggedright\textbf{\ALG@name~\thealgorithm} ##2\par}%
       \ifx\relax##1\relax % #1 is \relax
         \addcontentsline{loa}{algorithm}{\protect\numberline{\thealgorithm}##2}%
       \else % #1 is not \relax
         \addcontentsline{loa}{algorithm}{\protect\numberline{\thealgorithm}##1}%
       \fi
       \kern2pt\hrule\kern2pt
     }
  }{% \end{breakablealgorithm}
     \kern2pt\hrule\relax% \@fs@post for \@fs@ruled
   \end{center}
  }
\makeatother
\renewcommand{\algorithmicrequire}{\textbf{Input:}}  % Use Input in the format of Algorithm
\renewcommand{\algorithmicensure}{\textbf{Output:}} % Use Output in the format of Algorithm
\usepackage{longtable}

\usepackage{float}
\definecolor{gray}{rgb}{0.5,0.5,0.5}
\definecolor{dkgreen}{rgb}{.068,.578,.068}
\definecolor{dkpurple}{rgb}{.320,.064,.680}

% set Matlab styles
\lstset{
   language=Matlab,
   numbers=left,
   keywords={break,case,catch,continue,else,elseif,end,for,function,
      global,if,otherwise,persistent,return,switch,try,while},
   basicstyle=\ttfamily,
   keywordstyle=\color{blue}\bfseries,
   commentstyle=\color{dkgreen},
   stringstyle=\color{dkpurple},
   backgroundcolor=\color{white},
   tabsize=4,
   showspaces=false,
   showstringspaces=false,
}

\begin{document}
\CJKfamily{zhkai}


\begin{center}
    \textbf{作业二}\\
    \textbf{姓名 胡毅翔 ~~ 学号 PB18000290 ~~ 日期 2021年5月29日}\\
\end{center}

\begin{center}
    \fbox{
        \begin{minipage}{40em}
            \vspace{5cm}
            \hspace{20cm}
        \end{minipage}}
\end{center}
\vspace{1cm}

\begin{enumerate}
    \item[第一题] 本题考虑对于定义在 $[-1,1]$ 上的一个光滑函数 $f(x)$ 的三次样条插值的使用。下面 所说的误差都是指绝对误差。
    \begin{enumerate}\item(10分)仿照课堂笔记或课本推导出关于额外给定边界点处(即-1和1)三
    次样条插值多项式的一次导数值时其在各插值点上的二次导数值应该满足的
    线性方程组。请给出推导过程。
    \par 解:
    \par 记 $S(x)$ 在区间 $\left[x_{i}, x_{i+1}\right]$ 上的表达式为 $S_{i}(x), S(x)$ 是三次多项式, $S^{\prime \prime}(x)$ 是 线性函数, 用插值点 $\left\{\left(x_{i}, S^{\prime \prime}\left(x_{i}\right)\right),\left(x_{i+1}, S^{\prime \prime}\left(x_{i+1}\right)\right)\right\}$ 作线性插值, 记 $S^{\prime \prime}\left(x_{i}\right)=M_{i}$,
    $S^{\prime \prime}\left(x_{i+1}\right)=M_{i+1}$
    $$
    S_{i}^{\prime \prime}(x)=\frac{x-x_{i+1}}{x_{i}-x_{i+1}} M_{i}+\frac{x-x_{i}}{x_{i+1}-x_{i}} M_{i+1}, \quad x_{i} \leqslant x \leqslant x_{i+1}
    $$
    对 $S^{\prime \prime}(x)$ 积分两次, 记 $h_{i}=x_{i+1}-x_{i}$,
    $$
    \begin{aligned}
    S(x) &=S_{i}(x)=\frac{\left(x_{i+1}-x\right)^{3}}{6 h_{i}} M_{i}+\frac{\left(x-x_{i}\right)^{3}}{6 h_{i}} M_{i+1}+c x+d \\
    &=\frac{\left(x_{i+1}-x\right)^{3}}{6 h_{i}} M_{i}+\frac{\left(x-x_{i}\right)^{3}}{6 h_{i}} M_{i+1}+C\left(x_{i+1}-x\right)+D\left(x-x_{i}\right)
    \end{aligned}
    $$
    将 $S\left(x_{i}\right)=y_{i}, S\left(x_{i+1}\right)=y_{i+1}$ 代入上式解出
    $$
    C=\frac{y_{i}}{h_{i}}-\frac{h_{i} M_{i}}{6}, \quad D=\frac{y_{i+1}}{h_{i}}-\frac{h_{i} M_{i+1}}{6}
    $$
    \begin{equation}\label{eq1}
    \begin{aligned}
S(x)=& \frac{\left(x_{i+1}-x\right)^{3} M_{i}+\left(x-x_{i}\right)^{3} M_{i+1}}{6 h_{i}}+\frac{\left(x_{i+1}-x\right) y_{i}+\left(x-x_{i}\right) y_{i+1}}{h_{i}} \\
&-\frac{h_{i}}{6}\left[\left(x_{i+1}-x\right) M_{i}+\left(x-x_{i}\right) M_{i+1}\right], \quad x \in\left[x_{i}, x_{i+1}\right]
\end{aligned}\end{equation}

在内结点 $x_{i}$, 由 $S_{i}^{\prime}\left(x_{i}\right)=S_{i-1}^{\prime}\left(x_{i}\right)$ 可得到
\begin{equation}\label{eq2}
f\left(x_{i}, x_{i+1}\right)-\frac{h_{i}}{3} M_{i}-\frac{h_{i}}{6} M_{i+1}=f\left(x_{i-1}, x_{i}\right)+\frac{h_{i-1}}{6} M_{i-1}+\frac{h_{i-1}}{3} M_{i}
\end{equation}
整理后得到
$$
\mu_{i} M_{i-1}+2 M_{i}+\lambda_{i} M_{i+1}=d_{i}, \quad i=1,2, \cdots, n-1
$$
其中
$$
\begin{array}{c}
\lambda_{i}=\frac{h_{i}}{h_{i}+h_{i-1}}, \quad \mu_{i}=1-\lambda_{i} \\
d_{i}=\frac{6}{h_{i}+h_{i-1}}\left(\frac{y_{i+1}-y_{i}}{h_{i}}-\frac{y_{i}-y_{i-1}}{h_{i-1}}\right)=6 f\left(x_{i-1}, x_{i}, x_{i+1}\right)
\end{array}
$$
式 \ref{eq2} 称为样条插值的 $M$ 关系方程组,解方程组 \ref{eq2} 得到 $\left\{M_{i}, i=\right.$ $\left.1,2, \cdots, M_{n-1}\right\}$, 再加上两个端点条件, 满足端点条件的样条插值函数 $S(x)$ 在 $\left[x_{i},\right.$, $\left.x_{i+1}\right]$ 上的表达就是式 \ref{eq1}.
\par 给定 $S^{\prime}\left(x_{0}\right)=m_{0}, S^{\prime}\left(x_{n}\right)=m_{n}$ 的值, 将 $S^{\prime}\left(x_{0}\right)=m_{0}, S^{\prime}\left(x_{n}\right)=m_{n}$ 的值分
别代入 $S^{\prime}(x)$ 在 $\left[x_{0}, x_{1}\right],\left[x_{n-1}, x_{n}\right]$ 中的表达式, 得到另外两个方程:
$$
\begin{array}{c}
2 M_{0}+M_{1}=\frac{6}{h_{0}}\left[f\left[x_{0}, x_{1}\right]-m_{0}\right]=d_{0} \\
M_{n-1}+2 M_{n}=\frac{6}{h_{n-1}}\left[m_{n}-f\left[x_{n-1}, x_{n}\right]\right]=d_{n}
\end{array}
$$
得到 $n+1$ 个未知量, $n+1$ 个方程组
$$
\left[\begin{array}{cccccc}
2 & 1 & & & & \\
\mu _{1} & 2 & \lambda_{1} & & & \\
& \mu _{2} & 2 & \lambda_{2} & & \\
& & \ddots & \ddots & \ddots & \\
& & & \mu _{n-2} & 2 & \lambda_{n-1} \\
& & & & 1 & 2
\end{array}\right]\left[\begin{array}{c}
M_{0} \\
M_{1} \\
M_{2} \\
\vdots \\
M_{n-1} \\
M_{n}
\end{array}\right]=\left[\begin{array}{c}
d_{0} \\
d_{1} \\
d_{2} \\
\vdots \\
d_{n-1} \\
d_{n}
\end{array}\right]
$$
\item(10分)令三次样条插值多项式在一1和1处的导数为0, 用\textbf{Matlab}基于上
    一问中的结果使用 $n=2^{4}$ 个子区间插值一个定义在 $[-1,1]$ 上的函数 $f(x)=$ $\sin \left(4 x^{2}\right)+\sin ^{2}(4 x)$ 并使用semilogy图通过在2000个等距点上取真实值画出你 构造的三次样条插值的逐点误差。
    \item(15分)使用不同的 $n$, 令 $n=2^{4}, 2^{5}, \ldots, 2^{10}$ 重复上一问,取关于不同 $n$ 的2000个
    等距点上的误差的最大值,用loglog图描述插值区
    间上最大误差值随 $n$ 变化
    的情况(即横轴是 $n$ )。
    \item(15分)针对周期边界条件,即假设三次样条函数满足 $S^{\prime}(-1)=S^{\prime}(1)$ 和 $S^{\prime \prime}(-1)=$ $S^{\prime \prime}(1)$, 重复完成上面三问中的要求。
         

\end{enumerate}

\end{enumerate}




\end{document}
